% !TEX program = pdflatex
\documentclass{article}
\input{preamble/preamble}
\input{preamble/preamble_math}

\pagenumbering{gobble}

\begin{document}

\title{\textbf{Project Proposal}}
\date{\today}
\author{
Michael Alvarino maa2282\\
Colby Wise cjw2165
}

\maketitle

\section{Description}

\paragraph{Goal} The most popular route suggestion engines minimize the expected time to a destination based on maximizing speed while minimizing distance. These models use real-time data as reported by users to determine congestion, but surprisingly do not incorporate a predictive element of traffic congestion. Our first goal is to use taxi data as a proxy for traffic in Manhattan. This is the first step in creating a better route suggestion engine.

\paragraph{Data} In 2013 the NYC Taxi and Limousine Commission (TLC) began publicly releasing detailed yellow taxi, green taxi, and limousine ride information. Each data point included date, time, GPS location of pickup and dropoff, trip distance, payment type, and trip cost. We will be bucketing that data into different neighborhoods, with the precision of the neighborhood specification to be determined.

\section{Proposal}

\paragraph{Modeling}

Our goal is to discover a function \\
The gaussian process regression is a supervised learning technique that provides a mapping function between input and output data. It consideres pairs of input and noisy output data D and an unknown mapping function f(t) where t are timesteps. According to this definition the function f(t) is a set of random variables, where any subset has a joint gaussian distribution. \\
To model our larger goal, we intend to represent a series of routing decisions with a Pittman-Yor process (simlar to a Dirichlet process) and similar to what is done in \href{http://web.mit.edu/jaillet/www/general/IAT2012.pdf}{this} research paper. Because our data is limited to pickup and dropoff information we would predict routes in the form of borough to borough trips rather than street level granularity.

\paragraph{Inference}

\paragraph{Criticism}

Given our model, which predicts traffic from taxi data, we would compare against another source of travel time predictions. One such source is Google Maps, whose estimations we would use as a baseline of comparison.


\bibliographystyle{apa}
\bibliography{your_bib_file.bib}

\end{document}
