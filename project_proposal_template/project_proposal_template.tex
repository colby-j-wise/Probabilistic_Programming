% !TEX program = pdflatex
\documentclass{article}
\usepackage{listings}
\usepackage{tasks}
\usepackage{graphicx}
\usepackage{hyperref}

\pagenumbering{gobble}

\usepackage[                                                                       
 paper  = letterpaper,                                                            
 left   = 1.5in,                                                                 
 right  = 1.5in,                                                                 
 top    = 1.0in,                                                                  
 bottom = 1.0in,                                                                  
 ]{geometry}

\hypersetup{
    colorlinks=true,
    linkcolor=blue,
    filecolor=magenta,      
    urlcolor=magenta,
}

\begin{document}

\title{\textbf{Project Proposal}}
\date{\today}
\author{
Michael Alvarino maa2282\\
Colby Wise cjw2165
}

\maketitle

\section{Description}

\paragraph{Goal} The most popular route suggestion engines minimize the expected time to a destination based on maximizing speed while minimizing distance. These models use real-time data as reported by users to determine congestion, but have no predictive element of future traffic. Our first goal is to use taxi data as a proxy for traffic in Manhattan. This is the first step in creating a better route suggestion engine.

\paragraph{Data} We will be using the Taxi and Limousine Commission's yellow and green cab data. This includes date, time, GPS location of pickup and drop off, trip distance, payment type, and trip cost. We will bucket that data into different neighborhoods, with the granularity to be determined.

\section{Proposal}

\paragraph{Modeling}

As part of Box's Loop will have to experiment with a few temporal-spatial models. we will first implement a Gaussian Process Regression. It considers pairs of input and noisy output data $D$ and an unknown mapping function $f(t)$ where $t$ are time steps. This function can be viewed as a covariance function with unknown parameters $\theta$. The general nature of GP's means this can be solved in closed form using maximum likelihood estimation. Within the Gaussian Process we will compare model inference using different Kernel functions. \\
\textbf{Ref:} 
\href{http://www.synchromedia.ca/system/files/1570265950.pdf}{Gaussian Process Regression based Traffic Modeling}\\
\ \newline
To model our larger goal, we intend to represent a series of routing decisions with a Pittman-Yor process (a Dirichlet process with two parameters) similar to what is done in the research paper referenced below. Because our data is limited to pickup and drop-off information we would predict routes in the form of borough to borough trips rather than street level.\\
\textbf{Ref:} 
\href{http://web.mit.edu/jaillet/www/general/IAT2012.pdf}{Hierarchical Bayesian Approach to Modeling Urban Traffic}

\paragraph{Inference}
For the Gaussian Process model we can use gradient-based inference methods to solve for the posterior by maximizing the log-likelihood function. Depending on the complexity of the model we may use other techniques for large-scale inference like those referenced in the below NIPS paper. \textit{Scalable Inference for Gaussian Process Models with Black-Box Likelihoods}\\
\textbf{Ref:} 
\href{https://papers.nips.cc/paper/5665-scalable-inference-for-gaussian-process-models-with-black-box-likelihoods.pdf}{Scalable Inference for Gaussian Process Models}

\paragraph{Criticism}
Given our model, which predicts traffic from taxi data, we would compare against another sources of travel time predictions. One such source is Google Maps, whose estimations we would use as a baseline of comparison.


\bibliographystyle{apa}
\bibliography{your_bib_file.bib}

\end{document}
